\hypertarget{laurel-hardy}{%
\section{Laurel \& Hardy}\label{laurel-hardy}}

\begin{figure}[!ht]
  \begin{adjustwidth}{-\oddsidemargin-1in}{-\rightmargin}
    \centering
    \includegraphics[trim={0 7cm 0 2cm,}, clip, width=\paperwidth]{./S01/img/3/laurel-and-hardy.png}
    \caption{Laurel & Hardy\label{fig:laurel-hardy}}
  \end{adjustwidth}
\end{figure}

No apartamento de Joey e Chandler, enquanto ensaiam para a audição, um
poster de \emph{Laurel \& Hardy} aparece na sala de estar. Esses
personagens são conhecidos no Brasil como \emph{O Gordo e o Magro} e
fizeram cerca de 106 filmes juntos, entre curtas-metragem sonoros e
mudos e filmes.

\hypertarget{referuxeancias}{%
\subsection{Referências}\label{referuxeancias}}

\begin{itemize}
\tightlist
\item
  \sloppy Site oficial. \url{http://www.laurel-and-hardy.com/}
\item
  \sloppy Wikipédia. \url{https://pt.wikipedia.org/wiki/Laurel_%26_Hardy}
\end{itemize}

\hypertarget{grand-jury-secrets}{%
\section{Grand Jury Secrets}\label{grand-jury-secrets}}

\begin{figure}[!ht]
  \begin{adjustwidth}{-\oddsidemargin-1in}{-\rightmargin}
    \centering
    \includegraphics[trim={0 6cm 0 3cm,}, clip, width=\paperwidth]{./S01/img/3/grand-jury-secrets.png}
    \caption{Grand Jury Secrets\label{fig:grand-jury-secrets}}
  \end{adjustwidth}
\end{figure}

\emph{Grand Jury Secrets} (1939) é um filme de drama e mistério
americano. Conta a história de um repórter inescrupuloso que obtém
notícias através de um rádio escondido na sala de um júri, onde seu
irmão trabalha como advogado. O repórter é sequestrado e escapa devido
sua capacidade de transmitir sua situação.

\begin{figure}
  \centering
  \begin{tikzpicture}
    \node [inner sep=0pt] at (0,0) {
      \includegraphics[width=0.4\textwidth,keepaspectratio]{./S01/img/3/grand-jury-secrets-poster.jpg}
    };
    \draw [white, rounded corners=\ClipSep, line width=\ClipSep]
    (current bounding box.north west) --
    (current bounding box.north east) --
    (current bounding box.south east) --
    (current bounding box.south west) -- cycle
    ;
    \end{tikzpicture}
    \caption{Grand Jury Secrets poster\label{fig:grand-jury-secrets-poster}}
\end{figure}

\hypertarget{referuxeancias-1}{%
\subsection{Referências}\label{referuxeancias-1}}

\begin{itemize}
\tightlist
\item
  \sloppy BFI. \url{https://www.bfi.org.uk/films-tv-people/4ce2b6ab720c8}
\item
  \sloppy Loving The Classics. \url{https://www.lovingtheclassics.com/by-title/g/grand-jury-secrets-1939.html}
\item
  \sloppy IMDB. \url{https://www.imdb.com/title/tt0031390/}
\end{itemize}

\hypertarget{there-was-a-crooked-man}{%
\section{There was a crooked
man\ldots{}}\label{there-was-a-crooked-man}}

\begin{figure}[!ht]
  \begin{adjustwidth}{-\oddsidemargin-1in}{-\rightmargin}
    \centering
    \includegraphics[trim={0 7cm 0 2cm,}, clip, width=\paperwidth]{./S01/img/3/crooked-man.png}
    \caption{There was a crooked man\label{fig:there-was-a-crooked-man}}
  \end{adjustwidth}
\end{figure}

\begin{tcolorbox}[enhanced,center upper,
    drop fuzzy shadow southeast, boxrule=0.3pt,
    lower separated=false,
    colframe=black!30!dialogoBorder,colback=white]
\begin{minipage}[c]{0.14\linewidth}
  \raisebox{\dimexpr-\height+\ht\strutbox\relax}{
    \includegraphics[width=1.5cm]{./assets/img/phoebe.png}
  }
   & \centering \scriptsize{Phoebe}
\end{minipage}
\hspace{.1mm}
\begin{minipage}[c]{0.8\linewidth}
  \textbf{- From the nursery rhyme. 'There was a crooked man, Who had a crooked smile, Who lived in a shoe, For a... while...'}\\
  - Da musiquinha. 'Havia um cara de sorriso torto. E morava no sapato, por um... tempo...
\end{minipage}
\end{tcolorbox}

Enquanto os amigos discutem o quanto Alan é legal, Phoebe o compara com
o homem no sapato, que, como Joey notou, tem um sorisso torto. A
``musiquinha'' ou canção de ninar é \emph{There Was a Crooked Man} (C.
1840) de origem desconhecida.

Letra original e tradução da canção:

\bigskip
\begin{tcolorbox}[enhanced,
    drop fuzzy shadow southeast, boxrule=0.3pt,
    lower separated=false, sidebyside, sidebyside align=top,
    halign=flush right, halign lower=left,
    colframe=black!30!dialogoBorder,colback=musicaBg]
\includegraphics[width=0.4cm]{./assets/img/icon-music.png}\\
There was a crooked man, and he walked a crooked mile.\\He found a crooked sixpence upon a crooked stile.\\He bought a crooked cat, which caught a crooked mouse,\\And they all lived together in a little crooked house.\\
\tcblower
\includegraphics[width=0.4cm]{./assets/img/icon-language.png}\\
Havia um homem torto, e ele andou em uma estrada torta\\Ele encontrou uma moeda torta sobre um rebordo torto\\Ele comprou um gato torto, que pegou um rato torto\\E eles todos viveram juntos em uma pequena casa torta\\
\end{tcolorbox}

\hypertarget{referuxeancias-2}{%
\subsection{Referências}\label{referuxeancias-2}}

\begin{itemize}
\tightlist
\item
  \sloppy Wikipédia. \url{https://en.wikipedia.org/wiki/There_Was_a_Crooked_Man}
\item
  \sloppy There Was A Crooked Man Nursery Rhyme - YouTube. \url{https://www.youtube.com/watch?v=WqyUOlz_6i4}
\item
  Alchin, Linda Kathryn. \emph{Secret History of Nursery Rhymes.} Linda
  Alchin, 2010.
\end{itemize}

\hypertarget{david-hasselhoff}{%
\section{David Hasselhoff}\label{david-hasselhoff}}

\begin{figure}[!ht]
  \begin{adjustwidth}{-\oddsidemargin-1in}{-\rightmargin}
    \centering
    \includegraphics[trim={0 6cm 0 2cm,}, clip, width=\paperwidth]{./S01/img/3/david-hasselhoff.png}
    \caption{David Hasselhoff\label{fig:david-hasselhoff}}
  \end{adjustwidth}
\end{figure}

\begin{tcolorbox}[enhanced,center upper,
    drop fuzzy shadow southeast, boxrule=0.3pt,
    lower separated=false,
    colframe=black!30!dialogoBorder,colback=white]
\begin{minipage}[c]{0.14\linewidth}
  \raisebox{\dimexpr-\height+\ht\strutbox\relax}{
    \includegraphics[width=1.5cm]{./assets/img/chandler.png}
  }
   & \centering \scriptsize{Chandler}
\end{minipage}
\hspace{.1mm}
\begin{minipage}[c]{0.8\linewidth}
  \textbf{- I'd marry him just for his David Hasselhoff impression alone.}\\
  - Eu casaria por causa da imitação de David Hasselholff.
\end{minipage}
\end{tcolorbox}

\emph{David Hasselhoff} é ator, cantor, produtor e empresário
norte-americano. Um de seus grandes papeis de sucesso foi o salva-vidas
\emph{Mitch Buchannon} em \emph{Baywatch} (1989 - 2001), conhecida no
Brasil como \emph{S.O.S. Malibu}, série que acaba sendo conhecida como a
preferida de Joey no episódio
\textbf{\textcolor{ocre}{S03E06 - Aquele do flashback}}, em que ele
apresenta a série a Chandler.

\begin{figure}
  \centering
  \begin{tikzpicture}
    \node [inner sep=0pt] at (0,0) {
      \includegraphics[width=0.8\textwidth,keepaspectratio]{./S01/img/3/david-hasselhoff-baywatch.jpg}
    };
    \draw [white, rounded corners=\ClipSep, line width=\ClipSep]
    (current bounding box.north west) --
    (current bounding box.north east) --
    (current bounding box.south east) --
    (current bounding box.south west) -- cycle
    ;
    \end{tikzpicture}
    \caption{David Hasselhoff em Baywatch\label{fig:david-hasselhoff-em-baywatch}}
\end{figure}

\hypertarget{referuxeancias-3}{%
\subsection{Referências}\label{referuxeancias-3}}

\begin{itemize}
\tightlist
\item
  \sloppy IMDB. \url{https://www.imdb.com/name/nm0001327/}
\item
  \sloppy Site oficial Baywatch. \url{https://www.baywatch.com/}
\end{itemize}

\hypertarget{bugs-bunny}{%
\section{Bugs Bunny}\label{bugs-bunny}}

\begin{figure}[!ht]
  \begin{adjustwidth}{-\oddsidemargin-1in}{-\rightmargin}
    \centering
    \includegraphics[trim={0 7cm 0 2cm,}, clip, width=\paperwidth]{./S01/img/3/bugs-bunny.png}
    \caption{Bugs Bunny\label{fig:bugs-bunny}}
  \end{adjustwidth}
\end{figure}

\begin{tcolorbox}[enhanced,center upper,
    drop fuzzy shadow southeast, boxrule=0.3pt,
    lower separated=false,
    colframe=black!30!dialogoBorder,colback=white]
\begin{minipage}[c]{0.14\linewidth}
  \raisebox{\dimexpr-\height+\ht\strutbox\relax}{
    \includegraphics[width=1.5cm]{./assets/img/ross.png}
  }
   & \centering \scriptsize{Ross}
\end{minipage}
\hspace{.1mm}
\begin{minipage}[c]{0.8\linewidth}
  \textbf{- He was like that Bugs Bunny cartoon where Bugs is playing all the positions.}\\
  - Ele parecia o Pernalonga no desenho em que jogava em todas as posições.
\end{minipage}
\end{tcolorbox}

Enquanto explica como Alan joga bem \emph{softball}, Ross menciona
\emph{Bugs Bunny}, conhecido no Brasil como \emph{Pernalonga},
personagem da \emph{Looney Tunes}. O episódio citado por Ross é o
\emph{Baseball Bugs} (1946).

\begin{figure}
  \centering
  \begin{tikzpicture}
    \node [inner sep=0pt] at (0,0) {
      \includegraphics[width=0.4\textwidth,keepaspectratio]{./S01/img/3/baseball-bugs.jpg}
    };
    \draw [white, rounded corners=\ClipSep, line width=\ClipSep]
    (current bounding box.north west) --
    (current bounding box.north east) --
    (current bounding box.south east) --
    (current bounding box.south west) -- cycle
    ;
    \end{tikzpicture}
    \caption{Baseball Bugs\label{fig:baseball-bugs}}
\end{figure}

\hypertarget{referuxeancias-4}{%
\subsection{Referências}\label{referuxeancias-4}}

\begin{itemize}
\tightlist
\item
  \sloppy Fandom Wiki - Looney Tunes. \url{https://looneytunes.fandom.com/wiki/Baseball_Bugs}
\item
  \sloppy IMDB. \url{https://www.imdb.com/title/tt0038333/}
\item
  \sloppy Episódio na SuperCartoons (Inglês). \url{https://www.supercartoons.net/cartoon/629/bugs-bunny-baseball-bugs.html}
\item
  \sloppy Diferenças entre softball e baseball (Inglês). \url{https://www.dummies.com/sports/fantasy-sports/fantasy-baseball/the-differences-between-softball-and-baseball/}
\end{itemize}

\hypertarget{lamb-chop}{%
\section{Lamb Chop}\label{lamb-chop}}

\begin{figure}[!ht]
  \begin{adjustwidth}{-\oddsidemargin-1in}{-\rightmargin}
    \centering
    \includegraphics[trim={0 6cm 0 3cm,}, clip, width=\paperwidth]{./S01/img/3/lamb-chop.png}
    \caption{Lamb Chop\label{fig:lamb-chop}}
  \end{adjustwidth}
\end{figure}

\begin{tcolorbox}[enhanced,center upper,
    drop fuzzy shadow southeast, boxrule=0.3pt,
    lower separated=false,
    colframe=black!30!dialogoBorder,colback=white]
\begin{minipage}[c]{0.14\linewidth}
  \raisebox{\dimexpr-\height+\ht\strutbox\relax}{
    \includegraphics[width=1.5cm]{./assets/img/chandler.png}
  }
   & \centering \scriptsize{Chandler}
\end{minipage}
\hspace{.1mm}
\begin{minipage}[c]{0.8\linewidth}
  \textbf{- If I had a sock on my hand for 30 years, it'd be talking too.}\\
  - Se eu usasse uma meia na mão por 30 anos, ela também ia falar.
\end{minipage}
\end{tcolorbox}

No apartamento de Monica os amigos assistem a \emph{Lamb Chop}, fantoche
criado por \emph{Shari Lewis} (1934-1998). Sua filha, \emph{Mallory
Lewis}, continuou seu legado e faz performances de \emph{Lamb Chop} até
hoje.

\begin{figure}
  \centering
  \begin{tikzpicture}
    \node [inner sep=0pt] at (0,0) {
      \includegraphics[width=0.7\textwidth,keepaspectratio]{./S01/img/3/shari-lewis.jpg}
    };
    \draw [white, rounded corners=\ClipSep, line width=\ClipSep]
    (current bounding box.north west) --
    (current bounding box.north east) --
    (current bounding box.south east) --
    (current bounding box.south west) -- cycle
    ;
    \end{tikzpicture}
    \caption{Shari Lewis\label{fig:shari-lewis}}
\end{figure}

\hypertarget{referuxeancias-5}{%
\subsection{Referências}\label{referuxeancias-5}}

\begin{itemize}
\tightlist
\item
  \sloppy Site oficial. \url{https://mallorylewisandlambchop.com/faqs/}
\end{itemize}
